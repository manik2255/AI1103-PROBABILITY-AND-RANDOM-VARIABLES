\documentclass[journal,12pt,twocolumn]{IEEEtran}

\usepackage{setspace}
\usepackage{gensymb}
\singlespacing
\usepackage[cmex10]{amsmath}

\usepackage{amsthm}

\usepackage{mathrsfs}
\usepackage{txfonts}
\usepackage{stfloats}
\usepackage{bm}
\usepackage{cite}
\usepackage{cases}
\usepackage{subfig}

\usepackage{longtable}
\usepackage{multirow}

\usepackage{enumitem}
\usepackage{mathtools}
\usepackage{steinmetz}
\usepackage{tikz}
\usepackage{circuitikz}
\usepackage{verbatim}
\usepackage{tfrupee}
\usepackage[breaklinks=true]{hyperref}
\usepackage{graphicx}
\usepackage{tkz-euclide}

\usetikzlibrary{calc,math}
\usepackage{listings}
    \usepackage{color}                                            %%
    \usepackage{array}                                            %%
    \usepackage{longtable}                                        %%
    \usepackage{calc}                                             %%
    \usepackage{multirow}                                         %%
    \usepackage{hhline}                                           %%
    \usepackage{ifthen}                                           %%
    \usepackage{lscape}     
\usepackage{multicol}
\usepackage{chngcntr}

\DeclareMathOperator*{\Res}{Res}

\renewcommand\thesection{\arabic{section}}
\renewcommand\thesubsection{\thesection.\arabic{subsection}}
\renewcommand\thesubsubsection{\thesubsection.\arabic{subsubsection}}

\renewcommand\thesectiondis{\arabic{section}}
\renewcommand\thesubsectiondis{\thesectiondis.\arabic{subsection}}
\renewcommand\thesubsubsectiondis{\thesubsectiondis.\arabic{sub subsection}}


\hyphenation{optical networks semiconduc-tor}
\def\inputGnumericTable{}                                 %%

\lstset{
%language=C,
frame=single, 
breaklines=true,
columns=fullflexible
}
\date{March 2021}

\begin{document}

\newcommand{\BEQA}{\begin{eqnarray}}
\newcommand{\EEQA}{\end{eqnarray}}
\newcommand{\define}{\stackrel{\triangle}{=}}
\bibliographystyle{IEEEtran}
\raggedbottom
\setlength{\parindent}{0pt}
\providecommand{\mbf}{\mathbf}
\providecommand{\pr}[1]{\ensuremath{\Pr\left(#1\right)}}
\providecommand{\qfunc}[1]{\ensuremath{Q\left(#1\right)}}
\providecommand{\fn}[1]{\ensuremath{f\left(#1\right)}}
\providecommand{\e}[1]{\ensuremath{E\left(#1\right)}}
\providecommand{\sbrak}[1]{\ensuremath{{}\left[#1\right]}}
\providecommand{\lsbrak}[1]{\ensuremath{{}\left[#1\right.}}
\providecommand{\rsbrak}[1]{\ensuremath{{}\left.#1\right]}}
\providecommand{\brak}[1]{\ensuremath{\left(#1\right)}}
\providecommand{\lbrak}[1]{\ensuremath{\left(#1\right.}}
\providecommand{\rbrak}[1]{\ensuremath{\left.#1\right)}}
\providecommand{\cbrak}[1]{\ensuremath{\left\{#1\right\}}}
\providecommand{\lcbrak}[1]{\ensuremath{\left\{#1\right.}}
\providecommand{\rcbrak}[1]{\ensuremath{\left.#1\right\}}}
\theoremstyle{remark}
\newtheorem{rem}{Remark}
\newcommand{\sgn}{\mathop{\mathrm{sgn}}}
\providecommand{\abs}[1]{\vert#1\vert}
\providecommand{\res}[1]{\Res\displaylimits_{#1}} 
\providecommand{\norm}[1]{\lVert#1\rVert}
%\providecommand{\norm}[1]{\lVert#1\rVert}
\providecommand{\mtx}[1]{\mathbf{#1}}
\providecommand{\mean}[1]{E[ #1 ]}
\providecommand{\fourier}{\overset{\mathcal{F}}{ \rightleftharpoons}}
%\providecommand{\hilbert}{\overset{\mathcal{H}}{ \rightleftharpoons}}
\providecommand{\system}{\overset{\mathcal{H}}{ \longleftrightarrow}}
	%\newcommand{\solution}[2]{\textbf{Solution:}{#1}}
\newcommand{\solution}{\noindent \textbf{Solution: }}
\newcommand{\cosec}{\,\text{cosec}\,}
\providecommand{\dec}[2]{\ensuremath{\overset{#1}{\underset{#2}{\gtrless}}}}
\newcommand{\myvec}[1]{\ensuremath{\begin{pmatrix}#1\end{pmatrix}}}
\newcommand{\mydet}[1]{\ensuremath{\begin{vmatrix}#1\end{vmatrix}}}
\numberwithin{equation}{subsection}
\makeatletter
\vspace{3cm}
\title{ASSIGNMENT 6}
\author{MANIKANTA VALLEPU - AI20BTECH11014}
\maketitle
\newpage
\bigskip
\renewcommand{\thetable}{\theenumi}
Download all latex-tikz codes from 
\begin{lstlisting}
https://github.com/manik2255/AI1103-PROBABILITY-AND-RANDOM-VARIABLES/blob/main/ASSIGNMENT_6/ASSIGNMENT_6.tex
\end{lstlisting}

\section{GATE 2020 ST PROBLEM.43}
Let (X,Y) be a random vector such that, for any $y>0$, the conditional probability density function of X given $Y=y$ is $$f_{X|Y=y}(x)=ye^{-yx} \:,x>0. $$ If the marginal probability density function of Y is $$g(y)=ye^{-y}\:,y>0$$ then $E(Y|x=1)=$
\section{SOLUTION}
Given,
the conditional probability density function of X given $Y=y$,
\begin{align}
f_{X|Y=y}(x)=ye^{-yx} \:,x>0 \label{a}
\end{align}
and, the marginal probability density function of Y,
\begin{align}
g(y)=ye^{-y}\:,y>0 \label{b}
\end{align}
let the joint probability density function of (X,Y) be $f_{X,Y}(x,y)$.
We know that,
\begin{align}
f_{X|Y=y}(x)=\frac{f_{X,Y}(x,y)}{g(y)}  \label{c}
\end{align}
using \eqref{a} and \eqref{b} in \eqref{c},
\begin{align}
f_{X,Y}(x,y) =y^{2}e^{-y(x+1)} \:,x,y>0 \label{d}
\end{align}
let the marginal probability density function of X be $f_{X}(x)$,
as we know ,
\begin{align}
f_{X}(x)= \int_{0}^{\infty}{f_{X,Y}(x,y)}\,dy \label{e}
\end{align}
using \eqref{d} in \eqref{e},
\begin{align}
f_{X}(x) &=\int_{0}^{\infty}{y^{2}e^{-y(x+1)}}\,dy\\
&=\frac{2}{(x+1)^{3}} \:,x>0\label{f}
\end{align}
The conditional probability density function of Y given $X=x$ is given by,
\begin{align}
f_{Y|X=x}(y) =\frac{f_{X,Y}(x,y)}{f_{X}(x)} \label{g}
\end{align}
using \eqref{d} and \eqref{f} in \eqref{g},
\begin{align}
 f_{Y|X=x}(y) =\frac{y^{2}e^{-y(x+1)}(x+1)^{3}}{2} \:,x,y>0
\end{align}
The conditional probability density function of Y given $X=1$ is given by,
\begin{align}
 f_{Y|X=1}(y) =4y^{2}e^{-2y}  \:,y>0 \label{h}
\end{align}
We need to find $E(Y|X=1)$ which is given by,
\begin{align}
 E(Y|X=1) &= \int_{0}^{\infty}{yf_{Y|X=1}(y)}\,dy \label{i}
\end{align}
using \eqref{h} in \eqref{i},
\begin{align}
E(Y|X=1) &= \int_{0}^{\infty}{4y^{3}e^{-2y}}\,dy\\
  &= \left[\frac{-e^{-2y}(8y^{3} + 12y^{2} + 12y + 6)}{4}\right]_0^{\infty}\\
        &=\frac{3}{2}
\end{align}

\end{document}
